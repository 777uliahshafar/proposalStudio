\documentclass[handout]{beamer}

\input{preamble}

% -----------------------------------------------------------------------------------
% Notes
% -----------------------------------------------------------------------------------
\usepackage{pgfpages}
\setbeameroption{hide notes} % Only slides
%\setbeameroption{show only notes} % Only notes
%\setbeameroption{show notes on second screen=right} % Both

\setbeamercovered{transparent}
\setbeamertemplate{note page}{\pagecolor{yellow!5}\insertnote}\usepackage{palatino}

% -----------------------------------------------------------------------------------
% Supplement Package
% -----------------------------------------------------------------------------------
\usepackage{beamer_digiPH}
\usepackage{comment}
\usepackage{lipsum}

%\includeonlyframes{current4} % Uncomment this for speed compilation

\title{\#Preferensi Ruang di Kawasan Waterfront Parepare} \author{Muhammad Uliah Shafar}
\institute[Lehrstuhl]{
	Name des Lehrstuhls \\
	Name des Instituts/Fachbereichs \\
	Name der Universität, Ort
}
\mode<presentation>{\keywords{Schlüsselwörter durch Komma getrennt}}
\date[09.04.2018]{Datum der Präsentation, \zB 9. April 2018}


\begin{document}

\begin{withoutheadline}
%\begin{frame}
%	% Schriftgröße verkleinern
%	\small
%	% Absatzabstand einstellen
%	\setlength{\parskip}{0.75\baselineskip}
%
%	Dies ist die eLectures-Präsentationsvorlage der Virtuellen PH.
%
%	Die kommenden Folien in \LaTeX\ können und \textbf{\emph{sollen} Sie nach Wunsch adaptieren}, wir bitten Sie aber, sich alle genau anzusehen und durchzulesen! Denn:
%
%	Auf jeder Folie finden Sie Anregungen, Hilfestellungen und Hinweise, sowohl für die inhaltlich-didaktische Vorbereitung als auch die Präsentation online. Zumindest die vorhergehende Titelfolie sollte bitte auf jeden Fall in dieser Form Verwendung finden (Wiedererkennungswert!)
%
%	Bitte senden Sie Ihre überarbeitete PDF-Datei \textbf{vor dem gemeinsamen Testtermin} direkt an Ihre/n Co-Moderator/in. Er/sie bespricht sie mit Ihnen und lädt sie für die eLecture im Raum hoch. Der Kontakt zur Co-Moderation wird aktiv von dieser hergestellt.
%
%	\begin{block}{Zur didaktischen Planung der Stunde:}
%		Eine eLecture-Stunde geht oft sehr schnell vorbei! Gehen Sie daher bitte von max. \SI{75}{\percent} der Zeit für Ihre Präsentation aus, der Rest ist meist durch Einführung und Vorstellung, Interaktion, Rückfragen und Abschluss belegt. Je interaktiver Sie die eLecture halten, desto besser: besprechen Sie sich mit Ihrer CoModeration!
%	\end{block}
%\end{frame}

\begin{frame}
	%\maketitle
	% \maketitle funktioniert auch im Article-Modus
	 \titlepage
\end{frame}
\end{withoutheadline}

\miniframesoff
%\begin{frame}[label=inhalt]{Overview}
%	\tableofcontents
%\end{frame}
\miniframeson

\section{Pendahuluan}

\begin{frame}{Judul}

{\textbf{\large Desain Tepi Laut berdasarkan Preferensi Pengunjung}}
\vfill
\begin{description}
\item[\textbf{Desain}] artinya proses perancangan suatu objek.\\
\item[\textbf{Tepi Laut}] adalah kawasan dinamis yang berbatasan dengan air \citep{rizky2014}.\\
\item[\textbf{Preferensi}] adalah kecenderungan untuk memilih sesuatu yang disukai daripada yang lain \citep{devysandra2012}.
\end{description}

\end{frame}

\begin{frame}[t]{Kota Parepare}
    Parepare adalah salah satu kota administratif di Sulawesi Selatan. Saat ini, perencanaan kota mengarah pada pariwisata. Hal ini dikarenakan oleh beberapa alasan, diantaranya:
\medskip
    \begin{itemize}
        \item Parepare merupakan kota dengan lokasi strategis.\\ \pause
        \item Parepare telah memiliki ikon pariwisata yakni Patung BJ Habibie, Presiden ketiga Indonesia.\\ \pause
        \item Perkembangan kota yang sangat cepat dari berbagai aspek.\\ \pause
        \item Kota memiliki tepi laut sebagai bagian dari kota.
    \end{itemize}
\end{frame}

\begin{frame}{Objek Pariwisata Kota}
    Menurut \cite{junaid2016}, pengembangan pariwisata di kota Parepare berdasarkan konsep \textit{urban tourism} dan \textit{shopping tourism}. \pause

    Parepare cukup terkenal dengan beberapa objek wisata yang ada di pusat kota. Sejumlah faktor yang mendukung konsep pariwisata \textit{shopping toursim} seperti adanya pusat perbelanjaan:

\begin{enumerate}
    \item Pakaian bekas(Pasar Senggol).\\
    \item Barang-barang bekas(Carlos).\\
    \item Makanan-makanan khas bugis.
\end{enumerate}
   Tempat tersebut lokasinya berada tidak jauh dari tepi laut senggol.

\end{frame}

\begin{frame}{Ikon Pariwisata}
    Kota Toraja memiliki ikon tongkonan, kabupaten bulukumba memiliki perahu pinisi, sedangkan Parepare memiliki ikon patung Habibie dan Ainun.
\medskip
    Ikon ini melekat pada penataan kota Parepare, hal itu terbukti dengan pembangunan patung Habibie-Ainun, Balai Ainun, dan Museum Habibie-Ainun.
\medskip
    Ketiga objek bangunan tersebut berada di sekitar tepi laut senggol.
\end{frame}
\begin{frame}
\small Keberadaan objek bangunan tersebut menjadi faktor pendukung untuk peningkatan kawasan waterfront pantai senggol.

	\begin{center}
		\includegraphics[height=5cm, trim= 200 0 200 0, clip]{figures/patung}
        \linebreak
		{\tiny \textcolor{digiPH_darkblue}{Sumber: Detik.com, \href{https://creativecommons.org/licenses/by/3.0/at/}{CC BY}}}
	\end{center}
\end{frame}

\begin{frame}{Bangunan-bangunan sekitar objek}

\hfil\hfil\includegraphics[width=5cm]{figures/senggol}\newline
  \null\hfil\hfil\makebox[5cm]{Pasar Senggol}\newline
  \vfil
  \hfil\hfil\only<2->{\includegraphics[width=5cm]{figures/balai}}\hfil\hfil
   \only<3>{ \includegraphics[width=5cm]{figures/museum}}\newline
  \null\hfil\hfil\only<2->{\makebox[5cm]{Balai Habibie Ainun}}
    \hfil\hfil\only<3>{\makebox[5cm]{Museum Habibie Ainun}}

\end{frame}
\begin{frame}[t]

\hfil\hfil\includegraphics[width=5cm]{figures/masjid}\newline
  \null\hfil\hfil\makebox[5cm]{Masjid}\newline
  \vfil
  \hfil\hfil\only<2->{\includegraphics[width=5cm]{figures/hanstom}}\hfil\hfil
    \only<3>{\includegraphics[width=5cm]{figures/pelabuhan}}\newline
  \null\hfil\hfil\only<2->{\makebox[5cm]{Hanstom}}
    \hfil\hfil\only<3>{\makebox[5cm]{Pelabuhan}}

\end{frame}

\begin{frame}{Latar Belakang}

    Pada tahun 2011, kota Parepare memulai perencanaan pengembangan pasar senggol hingga kawasan tepi laut senggol.
    \vfill
    \only<2-> {Pengembangan ulang tepi laut senggol menghasilkan dua struktur yang berbeda.
    }
    \only<3-> {Pertama, area yang mempertahankan warisan tepi laut. Kedua, area yang menambahkan struktur baru.
    }
\end{frame}

\begin{frame}{Rumusan Masalah}
    Pada area yang mempertahankan warisan tepi laut memiliki pengunjung relatif tinggi, sedangkan area yang menambahkan struktur baru memiliki pengunjung yang sedikit.
    Maka, peneliti ingin {\Large mendesain kawasan} tepi laut yang menghadirkan struktur baru tanpa menghilangkan warisan yang telah ada.
\end{frame}

\begin{frame}{Tujuan}
    \large Desain ini bertujuan untuk {\Large memberi masukan} terhadap pengembangan ulang pada tepi laut senggol kota Parepare. Dengan menggabungkan struktur baru dan struktur lama, diharapkan desain ini dapat meningkatkan daya tarik.
\end{frame}


\section{Analisis}

\begin{frame}{Lokasi Perencanaan}
 	\begin{center}
		\includegraphics[width=\textwidth]{figures/lokzi.jpg}
		{\tiny \textcolor{digiPH_darkblue}{Sumber: Penulis, \href{https://creativecommons.org/licenses/by/3.0/at/}{CC BY}}}
	\end{center}

\end{frame}

\begin{frame}
\begin{itemize}

\item <1-> \cite{wang2020} menjelaskan preferensi penduduk kota terhadap urban space melalui sejumlah survei kepuasan terhadap kualitas fasilitas yang ada.

\item <2-> Menurut \cite{thomas2020}, preferensi suatu pengunjung didasari oleh penghilang stres dan rasa aman.

\item <3-> Berdasaarkan penelitian \citep{imansari2015} menghasilkan temuan bahwa preferensi masyarakat didasari oleh kecenderungan sebagai peneduh dan paru-paru kota.
\end{itemize}
\end{frame}

\begin{frame}{Area dengan Struktur Baru}
{\small atau} \large segmen 1
	\begin{center}
		\includegraphics[height=6cm]{figures/segmen1}

		{\tiny \textcolor{digiPH_darkblue}{Sumber: Penulis, \href{https://creativecommons.org/licenses/by/3.0/at/}{CC BY}}}
	\end{center}
\end{frame}

\begin{frame}{Area dengan Keberagaman Penggunaan}
{\small atau} \large segmen 2
	\begin{center}
		\includegraphics[height=6cm]{figures/segmen2}

		{\tiny \textcolor{digiPH_darkblue}{Sumber: Penulis, \href{https://creativecommons.org/licenses/by/3.0/at/}{CC BY}}}
	\end{center}
\end{frame}

\begin{frame}{Karakteristik Area dengan Struktur Baru}
	\begin{columns}
		\begin{column}{5cm}
			\small Memiliki footpath, penerangan, dan pembatas.
            \end{column}
		\begin{column}{5cm}
			\begin{center}
				\includegraphics[height=3cm]{figures/karakteristik_seg1}

				{\tiny \textcolor{digiPH_darkblue}{Sumber: Penulis, \url{pixabay.com}, CC-0}}
			\end{center}
		\end{column}

	\end{columns}\only<2->{
	\begin{columns}
		\begin{column}{5cm}
			\begin{center}
				\includegraphics[width=3cm]{figures/karakteristik2_seg1}

				{\tiny \textcolor{digiPH_darkblue}{Sumber:  Penulis, \url{pixabay.com}, CC-0}}
			\end{center}
		\end{column}
		\begin{column}{5cm}
			\begin{block}
				\small Squre tempat ruang publik.
			\end{block}
		\end{column}
	\end{columns}}
\end{frame}

\begin{frame}
	\small Lahan yang luas untuk parkir dan mobilisasi kendaraan.

	\begin{center}
		\includegraphics[height=5cm]{figures/karakteristik3_seg1}

		{\tiny \textcolor{digiPH_darkblue}{Sumber: Penulis, \href{https://creativecommons.org/licenses/by/3.0/at/}{CC BY}}}
	\end{center}
\end{frame}

\begin{frame}{Karakteristik Area dengan Warisan yang dipertahankan}
	\begin{columns}
		\begin{column}{5cm}
			\small Jalan Pedestrian digunakan sebagai tempat makan dan bersantai
            \end{column}
		\begin{column}{5cm}
			\begin{center}
				\includegraphics[height=3cm]{figures/karakteristik_seg2}

				{\tiny \textcolor{digiPH_darkblue}{Sumber: Penulis, \url{pixabay.com}, CC-0}}
			\end{center}
		\end{column}
	\end{columns}\only<2->{
	\begin{columns}
		\begin{column}{5cm}
			\begin{center}
				\includegraphics[width=3cm]{figures/karakteristik2_seg2}

				{\tiny \textcolor{digiPH_darkblue}{Sumber:  Penulis, \url{pixabay.com}, CC-0}}
			\end{center}
		\end{column}
		\begin{column}{5cm}
			\begin{block}
				\small Pedagang kaki lima di seputar area.
            \end{block}
		\end{column}
	\end{columns}}
\end{frame}

\begin{frame}
	\begin{columns}
		\begin{column}{5cm}
	\small Aktivitas berenang yang tercampur dengan aktivitas lain.
            \end{column}
		\begin{column}{5cm}
			\begin{center}
				\includegraphics[height=3cm]{figures/karakteristik3_seg2}

				{\tiny \textcolor{digiPH_darkblue}{Sumber: Penulis, \url{pixabay.com}, CC-0}}
			\end{center}
		\end{column}
	\end{columns}\only<2->{
	\begin{columns}
		\begin{column}{5cm}
			\begin{center}
				\includegraphics[width=3cm]{figures/karakteristik4_seg2}

				{\tiny \textcolor{digiPH_darkblue}{Sumber:  Penulis, \url{pixabay.com}, CC-0}}
			\end{center}
		\end{column}
		\begin{column}{5cm}
			\begin{block}
        \small Aktivitas memancing yang membutuhkan tempat khusus.
            \end{block}
		\end{column}
	\end{columns}}
\end{frame}

\begin{frame}{Deskripsi Pengunjung}
   Berdasarkan pengamatan pengunjung pada kawasan waterfront pantai senggol pada hari sabtu pagi hari 07.00-09.00 sebagai berikut :
\only<2->{\small{
   \begin{table}[htpb]
       \centering
       \caption{Karakter Pengunjung }
       \label{tab:label}
       \begin{tabular}{l l}
       Karakter & Jumlah(est) \\
\multicolumn{2}{c}{\bf Segmen 1}\\
        Bersantai sambil makan & $\pm$ 20 \\
        Pemancing & Tidak ada \\
        Berenang & Tidak ada\\
\multicolumn{2}{c}{\bf Segmen 2}\\
       Bersantai sambil makan & $\pm$ 30\\
       Berenang & $\pm$ 20 \\
       Pemancing & $\pm$ 8 \\
       \end{tabular}
   \end{table}}}
\only<3>{Dari hasil observasi awal ini, dapat disumpulkan bahwa kegiatan pada kawasan waterfront berfokus pada segmen 2. Sehingga pada kawasan segmen 1 mengalami sepi pengunjung.}
\end{frame}

\begin{comment}
\begin{frame}[t]{Deskripsi Segmen 1} \vspace{10pt}

\begin{table}[htpb]
    \centering
    \caption{Deskripsi Segmen 1}
    \label{tab:label}
    \begin{tabular}{c}
    \ldots
    \end{tabular}
\end{table}

\end{frame}

\begin{frame}[t]{Deskripsi Segmen 1} \vspace{10pt}

\begin{table}[htpb]
    \centering
    \caption{Deskripsi Segmen 1}
    \label{tab:label}
    \begin{tabular}{c}
    \ldots
    \end{tabular}
\end{table}

\end{frame}
\end{comment}

\begin{frame}{\large Dampak revitalisasi terhadap preferensi masayarakat}

Pada segmen 2 dimana pengunjung lebih cenderung berada, saat ini menikmati naungan dan lantai yang lebih bagus dibanding sebelumnya. Naungan ini juga memberikan batasan-batasan terhadap teritori PKL yang ada disana.
\vfill
Perbaikan pada segmen 2 ini meningkatkan keramaian pada tempat ini, peneliti menilai perlu peningkatan fasilitas untuk memadai keramaian tersebut.

\end{frame}

\begin{frame}{\large Dampak revitalisasi terhadap preferensi masayarakat}

Meskipun segmen 1 memiliki lebih sedikit pengunjung, namun ini dapat dimanfaatkan bagi mereka yang ingin menghindari keramaian.
\vfill
Peneliti juga menilai bahwa terdapat sejumlah kekurangan pada kawasan waterfront ini sebagai bagian dari ruang publik. Beberapa poin yang kurang diantaranya konektivitas, pedestrian way dan lain-lain.
\end{frame}

\begin{frame}[t]{\large Implementasi revitalisasi untuk mendukung preferensi pengunjung}

Secara keseluruhan, revitalisasi pada kawasan waterfront Pantai Senggol ini belum maksimal dalam mendukung preferensi pengunjung. Ada beberapa masalah terkait dengan dukungan kawasan ini terhadap preferensi pengunjung, diantaranya:
\begin{itemize}
    \item
    Pengunjung lebih meminati kawasan segmen 2 sehingga membuat segmen 1 sepi pengunjung. Dalam kata lain, secara kesuluran pengunjung pada kawasan pantai senggol tidak merata.
    \item Aktivitas berenang tidak memiliki fasilitas yang memadai, serta tempatnya yang tercampur dengan tempat makan juga menjadi perhatian.
    \item Sejumlah pemancing tidak memliki tempat khusus, kelompok ini memancing dimanapun yang mereka suka di sepanjang tepi laut.
\end{itemize}

\end{frame}

\begin{frame}[label=current]{Block Map}
 	\begin{center}
		\includegraphics[width=.9\textwidth]{figures/blockmap}
	\end{center}

\end{frame}



\begin{frame}[label=current]{Graph Axial Jalan}
 	\begin{center}
		\includegraphics[width=.9\textwidth]{figures/rod}
	\end{center}

\end{frame}

\begin{frame}[label=current]{Connectivity Axial}
\begin{columns}
\begin{column}{6cm}
 	\begin{center}
		\includegraphics[width=\textwidth]{figures/connectaxi}
        \end{center}
\end{column}
\begin{column}{4cm}
		{\small Grafik ini menunjukkan hubungan langsung node ke nodes lainnya atau lingkungan sekitar dan lainnya. Ruang axial yang paling banyak konektivitasnya terjadi pada Jalan Bau Massepe dekat lokasi peneli.}
\end{column}
\end{columns}
\end{frame}

\begin{frame}[label=current]{Integrated Axial}
\begin{columns}
\begin{column}{6cm}
 	\begin{center}
		\includegraphics[width=\textwidth]{figures/intgraxi}
        \end{center}
\end{column}
\begin{column}{4cm}
		{\small Integrasi biasanya mengindikasi kemungkinan jumlah orang di ruang tersebut. Lokasi penelitian menunjukkan bahwa integrasinya adalah sedang, artinya hanya membutuhkan sedikit belokan untuk mencapainya.}
\end{column}
\end{columns}
\end{frame}

\begin{frame}[label=current]{Choice Axial}
\begin{columns}
\begin{column}{6cm}
 	\begin{center}
		\includegraphics[width=\textwidth]{figures/choiceaxi}
        \end{center}
\end{column}
\begin{column}{4cm}
		{\small Grafik ini menunjukkan bahwa suatu tempat memiliki potensi untuk menghubungkan tempat asal dan tujuan lebih dekat. Sejumlah choice tersebar di pusat kota, dan dekat dari lokasi penelitian.  }
\end{column}
\end{columns}
\end{frame}

\begin{frame}[label=current]{Graph Visual Ruang}
 	\begin{center}
		\includegraphics[width=.9\textwidth]{figures/vis}
	\end{center}
\end{frame}

\begin{frame}[label=current]{Visual Ruang}
\begin{columns}
\begin{column}{6cm}
 	\begin{center}
		\includegraphics[width=\textwidth]{figures/connectvis}
        \end{center}
\end{column}
\begin{column}{4cm}
		{\small Grafik ini menjelaskan ruang yang mudah terlihat dari sejumlah tempat. Tepi laut menjelaskan tingkat visibilitasnya adalah sedang, artinya lokasi mudah terlihat dari berbagai tempat.}
\end{column}
\end{columns}
\end{frame}

\begin{frame}[label=current]{Step Depth Ruang}
\begin{columns}
\begin{column}{6cm}
 	\begin{center}
		\includegraphics[width=\textwidth]{figures/depthvis}
        \end{center}
\end{column}
\begin{column}{4cm}
		{\small Grafik Step Depth menggambarkan lokasi yang dapat terlihat dari suatu posisi. Lokasi Tepi pantai senggol merupakan area pantai yang visual terhadap laut sangat tinggi.}
\end{column}
\end{columns}
\end{frame}

\begin{frame}[label=current]{Agen Umum}
\begin{columns}
\begin{column}{6cm}
 	\begin{center}
		\includegraphics[width=\textwidth]{figures/agentumum}
        \end{center}
\end{column}
\begin{column}{4cm}
		{\small Grafik ini menunjukkan alur pergerakan pada model layout. Lokasi penelitian memiliki pergerakan rendah ke sedang.}
\end{column}
\end{columns}
\end{frame}

\begin{frame}[label=current]{Agen Terpilih}
\begin{columns}
\begin{column}{6cm}
 	\begin{center}
		\includegraphics[width=\textwidth]{figures/agenterpilih}
        \end{center}
\end{column}
\begin{column}{4cm}
		{\small Grafik ini menggambarkan apabila posisi pergerakan bermula dari tepi laut, maka kebanyakan menuju ke arah Lapangan Andi Makassau(tempat pergerakan paling tinggi).}
\end{column}
\end{columns}
\end{frame}

\begin{frame}[label=current4]{Analisis Tapak} \vspace{4pt}
 	\begin{center}
		\includegraphics[width=.75\textwidth]{figures/overallanalysis}
    \end{center}
\end{frame}

\begin{frame}[label=current4]{Analisis Tapak} \vspace{4pt}
 	\begin{center}
		\includegraphics[width=.9\textwidth]{figures/panalysis}
    \end{center}
\end{frame}


\begin{frame}[label=current4]{Zoning}
	\begin{columns}
		\begin{column}{4cm}
			\small Parkir diletakkan di area selatan karena pergerakan berdasarkan \textit{agent analysis} menuju ke selatan dan memusat di lapangan Andi Makassau. Zona menarik diletakkan di bagian utara karena banyak tempat keraimaian di utara tempat.
            \end{column}
		\begin{column}{6cm}
			\begin{center}
				\includegraphics[height=5cm]{figures/zoning}
			\end{center}
		\end{column}
	\end{columns}
\end{frame}

\begin{frame}[label=current4]{Sirkulasi}
	\begin{columns}
		\begin{column}{4cm}
			\small Selain tepi laut membutuhkan area komersial, lingkungan binaan juga harus menopang tempat ini. Jalur pedestrian diharapkan meningkatkan integrasi ruang.
            \end{column}
		\begin{column}{6cm}
			\begin{center}
				\includegraphics[height=5cm]{figures/jalur}
			\end{center}
		\end{column}
	\end{columns}
\end{frame}

\begin{frame}[label=current4]{Potongan} \vspace{4pt}
 	\begin{center}
		\includegraphics[width=\textwidth]{figures/pot1}
    \end{center}
\end{frame}


\section{Konsep 3d desain}

\begin{frame}[label=current]{Konsep Perencanaan}
Perencanaan pada tepi laut ini menggunakan konsep berkelanjutan.

Berdasarkan \cite{imperatives1987}, \textit{Sustainability development} adalah pengembangan yang memenuhi kebutuhan saat ini tanpa mempertaruhkan kemampuan dari generasi akan datang untuk memenuhi kebutuhannya.

\end{frame}

\begin{frame}[label=current]{Lokasi Perencanaan}
 	\begin{center}
		\includegraphics[width=.9\textwidth]{figures/atapak}
	\end{center}
\end{frame}

\begin{frame}[label=current]{Konsep Ide} \vspace{4pt}
 	\begin{center}
		\includegraphics[width=.9\textwidth]{figures/ide}
	\end{center}
\end{frame}

\begin{frame}[label=current]{Ide Desain}
 	\begin{center}
		\includegraphics[width=\textwidth]{figures/kon}
	\end{center}
\end{frame}


%--------------------------------------------------------------------------------------------
%
%--------------------------------------------------------------------------------------------

\begin{comment}
\begin{frame}{Lorem Ipsum}
\vspace{4pt}
	\begin{itemize}
\item \lipsum[10][1-2]
\item \lipsum[1][1-2]
	\end{itemize}

\end{frame}

\begin{frame}{Lorem Ipsum}
	\begin{center}
		\includegraphics[height=5.5cm]{figures/placeholder}

		{\tiny \textcolor{digiPH_darkblue}{\lipsum[1][1] \url{penulis}}}
	\end{center}
	\begin{center}
		\textbf{Lorem Lipsum} \lipsum[1][1]

	\end{center}
\end{frame}

\begin{frame}
	\begin{center}
		\includegraphics[height=5.5cm]{figures/placeholder}
		{\tiny \textcolor{digiPH_darkblue}{\lipsum[1][1] \url{penulis}}}
	\end{center}
	\begin{center}
		{\small\textbf{Lorem Lipsum} \lipsum[1][1]}
	\end{center}
\end{frame}


\section{Lorem}

\begin{frame}{Lorem Ipsum}
	\small \lipsum[1][1-2]
%	\tiny Weiteres dazu entnehmen Sie bitte dem Infoblatt: \url{http://www.virtuelle-ph.at/selbst-electures-abhalten/infoblatt}

%	\small Die Teilnehmenden werden über einen Disclaimer zu Beginn der eLecture darüber informiert:

	\begin{center}
		\includegraphics[height=5cm]{figures/placeholder}

		{\tiny \textcolor{digiPH_darkblue}{Sumber: Penulis, \href{https://creativecommons.org/licenses/by/3.0/at/}{CC BY}}}
	\end{center}
\end{frame}

%	\tiny Weiteres dazu entnehmen Sie bitte dem Infoblatt: \url{http://www.virtuelle-ph.at/selbst-electures-abhalten/infoblatt}

%	\small Die Teilnehmenden werden über einen Disclaimer zu Beginn der eLecture darüber informiert:

%-------------------------------------------------------------------------------------
\section{Lorem}

\begin{frame}{Highlight}
	\begin{center}
		\Large
		online immer lieber {\Huge mehr} als weniger\\
		und häufig wechseln!
	\end{center}
	\pause
	\begin{center}
		\Large
		Das bringt Abwechslung und Bewegung\\auf
        den Bildschirm und hilft,\\
        {\Huge Aufmerksamkeit} zu {\Huge halten}.
	\end{center}
\end{frame}

\begin{frame}{Kolom gambar dan teks}
	\begin{columns}
		\begin{column}{5cm}
			\begin{center}
				\includegraphics[height=6cm]{figures/chamaeleon_hochformat}

				{\tiny \textcolor{digiPH_darkblue}{Bildquelle: \url{pixabay.com}, CC-0}}
			\end{center}
		\end{column}
		\begin{column}{5cm}
			\begin{center}
				\Large
				Bitte mit Bedacht!
			\end{center}
			\pause
			\begin{center}
				Nicht mehr als ein bis zwei verschiedene Schriftfarben verwenden (Ruhe, bessere Lesbarkeit).
			\end{center}
			\begin{center}
				Tipp: Lieber durch Bilder Abwechslung schaffen!
			\end{center}
		\end{column}
	\end{columns}
\end{frame}

\begin{frame}
	%Gambar besar dengan penjelasan tanpa judul
	\begin{center}
		\includegraphics[height=6cm]{figures/kopfhoerer}

		{\tiny \textcolor{digiPH_darkblue}{Nicht die Bildquelle und die Lizenz vergessen! Bildquelle: \url{pixabay.com}, CC-0}}
	\end{center}
	\begin{center}
		Haben Sie Mut zur Vereinfachung und zu ungewohnter Positionierung und finden Sie interessante Ausschnitte.
	\end{center}
\end{frame}

\begin{frame}{Gambar besar dengan judul dan penjelasan}
	\begin{center}
		\includegraphics[height=5.5cm]{figures/placeholder}

		{\tiny \textcolor{digiPH_darkblue}{Bildquelle: \url{pixabay.com}, CC-0}}
	\end{center}
	\begin{center}
		Geschlechtsneutrales Formulieren ist uns wichtig: bei unseren eLectures ebenso. \textbf{In Schrift, Wort und Bild!}
	\end{center}
\end{frame}


\begin{frame}{Gambar dengan penjelasan sebelumnya}
	\small Ihre eLecture wird zum Nachsehen aufgezeichnet und veröffentlicht. Deshalb muss sie inhaltlich und urheberrechtlich korrekt sein!

	\tiny Weiteres dazu entnehmen Sie bitte dem Infoblatt: \url{http://www.virtuelle-ph.at/selbst-electures-abhalten/infoblatt}

	\small Die Teilnehmenden werden über einen Disclaimer zu Beginn der eLecture darüber informiert:

	\begin{center}
		\includegraphics[height=3.5cm]{figures/placeholder}

		{\tiny \textcolor{digiPH_darkblue}{Bildquelle: Lene Kieberl, \href{https://creativecommons.org/licenses/by/3.0/at/}{CC BY}}}
	\end{center}
\end{frame}


\begin{frame}[t]
	\begin{columns}
		\begin{column}{.33\textwidth}
			\begin{center}
				\includegraphics[width= \columnwidth]{figures/chamaeleon_hochformat}

				{\tiny \textcolor{digiPH_darkblue}{A}}
			\end{center}
		\end{column}
		\begin{column}{.33\textwidth}
			\begin{center}
				\includegraphics[width= \columnwidth]{figures/chamaeleon_hochformat}

				{\tiny \textcolor{digiPH_darkblue}{B}}
			\end{center}
		\end{column}
		\begin{column}{.33\textwidth}
			\begin{center}
				\includegraphics[width= \columnwidth]{figures/chamaeleon_hochformat}

				{\tiny \textcolor{digiPH_darkblue}{C}}
			\end{center}
		\end{column}
	\end{columns}
\end{frame}

\begin{frame}{Kolom dengan foto orang}
	\begin{columns}
		\begin{column}{5cm}
			\small besonders aber bei Minderjährigen bitte unbedingt abklären, ob diese bzw. ihre Erziehungsberechtigten mit der Veröffentlichung einverstanden sind (Recht am eigenen Bild).
		\end{column}
		\begin{column}{5cm}
			\begin{center}
				\includegraphics[height=2.5cm]{figures/placeholder}

				{\tiny \textcolor{digiPH_darkblue}{Bildquelle: Details Foto XY, \url{pixabay.com}, CC-0}}
			\end{center}
		\end{column}
	\end{columns}
	\begin{columns}
		\begin{column}{5cm}
			\begin{center}
				\includegraphics[width=5cm]{figures/placeholder}

				{\tiny \textcolor{digiPH_darkblue}{Bildquelle: Details Foto XY, \url{pixabay.com}, CC-0}}
			\end{center}
		\end{column}
		\begin{column}{5cm}
			\begin{block}{\small Tipp:}
				\scriptsize
				Als Hilfestellung können Sie vielleicht unseren \enquote{Schummelzettel OER} mit Tipps und Quellen für verwendbare Bilder nutzen, den Sie hier zum Download vorfinden:\\
				\url{http://www.virtuelle-ph.at/schummelzettel}
			\end{block}
		\end{column}
	\end{columns}
\end{frame}

\begin{frame}{Sie möchten Software/Seiten live vorzeigen?}
	% Schriftgröße verkleinern
	\scriptsize
	% Absatzabstand einstellen
	\setlength{\parskip}{0.75\baselineskip}

	Beachten Sie bitte, dass Sie beim Zeigen von längeren Ausschnitten einer \textbf{Programmoberfläche} (Vorzeigen von Abläufen die nicht mehr durch das Zitatrecht abgedeckt sind) gegebenenfalls die \textbf{Erlaubnis der Herstellerfirma einholen müssen}, sofern Sie nicht Urheber/in sind bzw. keine lizenzrechtliche Erlaubnis zur Vorführung vorliegt (mehr Info: \url{http://bit.ly/vphuhr}).

	\textbf{Diese Angabe kann Ihre Co-Moderation für Sie einblenden, bitte teilen Sie sie bei Testtermin oder via Email mit.}

	\begin{block}{\scriptsize Beispiel: Live-Demo \href{GIMP}{http://www.gimp.org/},GNU General Public License 29.01.2018}
		\centering
		\includegraphics[height=4cm]{figures/placeholder}
	\end{block}
\end{frame}

%\end{comment}
%\begin{comment}

% Hanya untuk Navigasi Miniframe
\miniframesoff
\begin{frame}<beamer>{}
	\begin{center}
		\ldots Manege frei für Ihr Knowhow und Ihre Folien!
	\end{center}
	\vspace{3cm}
	\tiny
	PS: Am Ende Ihrer eLecture weist Ihre Co-Moderation die Teilnehmenden noch auf weitere Angebote (\zB der aktuellen Online-Tagung) und die Social-Media-Präsenzen hin. Wir bitten Sie auch um Weiterverbreitung unseres (und Ihres!) Lernangebots. Selbst schon geliked? Wir freuen uns über Likes, Kommentare und natürlich Follower!
\end{frame}
\end{comment}

\begin{frame}<beamer>{}
\bibliographystyle{apalike}
{\tiny
\bibliography{biblio.bib}
}
\end{frame}

\end{document}
